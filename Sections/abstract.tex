\prefacesection{Abstract}

Organic solar cells are photovoltaic devices that use semiconducting plastics as the active layer rather than traditional inorganic materials such as Silicon.  Like any solar cell, their efficiency at producing electricity from sunlight is characterized by three parameters: their short-circuit current (Jsc), open-circuit voltage (Voc) and fill factor (FF).   While the factors that determine each of these parameters are well-understood for established solar technologies, this is not the case for organic solar cells.  The short-circuit current is much higher than we would expect given the strong attraction between electrons and holes in organic semiconductors that should lead to fast recombination, preventing the carriers from being collected as current.  In contrast, the open-circuit voltage is much lower than we would expect based on the traditional relationship between optical absorption and voltage in inorganic semiconductors.  Finally, the fill factor is highly variable from device to device and typically gets much worse as the cells are made thicker.

In this work we develop a novel and general framework for understanding the short-circuit current, open-circuit voltage and fill factor of organic solar cells.  The concept that turns out to unify all three aspects of device operation is the idea that electrons and holes move rapidly enough relative to their lifetimes to equilibrate with each other in the statistical mechanics sense before recombining.  Previously, it had been thought that such equilibration was impossible because of the low, macroscopic mobilities of charge carriers in organic solar cells.

We first show using Kinetic Monte Carlo simulations that the charge carrier mobility is 3-5 orders of magnitude higher on short length scales and immediately after light absorption by comparing simulated results to experimental terahertz spectroscopy data.  Combining this high mobility with experimental lifetime data fully rationalizes high charge carrier generation efficiency and also explains how carriers can live long enough to be affected by strong inhomogeneities in the energetic landscape of the solar cell, which also improves charge generation.  

Turning to Voc, we use the same concept of fast carrier motion relative to the recombination rate to show that recombination proceeds from an equilibrated population of Charge Transfer states.  This simplification permits us to develop an analytical understanding of the open-circuit voltage and explain numerous puzzling Voc trends that have been observed over the years.

Finally, we generalize our equilibrium result from open-circuit to explain the entire IV curve and use it to show how the low fill-factor of organic solar cells is not caused, as is often thought, by a voltage dependent carrier generation process but instead by the presence of dark charge carriers injected during device fabrication.  

Taken together, these results represent the first complete theory of organic solar cell operation.